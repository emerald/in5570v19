\begin{frame}[fragile]

\frametitle{Type Conformity Is Everywhere}

\begin{center}

When using an object where a particular type is expected,\\ the object
type must \textbf{\underline{conform to}} the expected type

\end{center}

\begin{itemize}

\item Conformity is checked at compile time; no runtime costs!

\end{itemize}

\begin{lstlisting}
const RandType <- typeobject RandType
  op next -> [seed : Integer]
end RandType

const RandClass <- class RandClass
  export op next -> [retval : Integer]
    retval <- 42
  end next
end rand

const main <- object main
  initially
    const r : RandType <- RandClass.create
  end initially
end main
\end{lstlisting}

\end{frame}


\begin{frame}[fragile]

\frametitle{Type Conformity: Definition}

\newcommand{\emlbr}{\ensuremath{\text{\texttt{[}}}}
\newcommand{\emrbr}{\ensuremath{\text{\texttt{]}}}}
\newcommand{\emrarr}{\ensuremath{\text{\texttt{->}}}}
\newcommand{\emsig}[3]{\ensuremath{#1\emlbr {#2}\emrbr\emrarr\emlbr {#3}\emrbr}}

A type $S$ conforms to a type $T$ iff for each operation

$$\emsig{o}{p^T_1, p^T_2, \ldots p^T_n}{r^T_1, r^T_2, \ldots,
r^T_m}\quad \text{in}\ T$$

there is a corresponding operation\footnote{Having the same name,
number of formal parameters, and results.}

$$\emsig{o}{p^S_1, p^S_2, \ldots p^S_n}{r^S_1, r^S_2, \ldots,
r^S_m}\quad \text{in}\ S$$

where

\begin{enumerate}

\item $p^T_i$ conforms to $p^S_i$, for all $i \in 1, 2, \ldots n$, and

\item $r^S_i$ conforms to $r^T_i$, for all $i \in 1, 2, \ldots n$

\end{enumerate}

\vspace{\fill}

\begin{center}

NB! Formal parameters conform one way, while results the other.

\end{center}

\end{frame}


\begin{frame}

\frametitle{Some Special Cases: Any and None}

\begin{center}

\lstinline{Any} and \lstinline{None} are special built-in types

\lstinline{None} is the type of the keyword (expression) \lstinline{nil}

\end{center}

\vspace{\fill}

They have the following interesting properties:

\begin{enumerate}

\item Everything conforms to \lstinline{Any}

\item \lstinline{None} conforms to anything

\item Nothing conforms to \lstinline{None}

\end{enumerate}

Notably, \lstinline{nil} conforms to \lstinline{Any}, and anything at all

\end{frame}


\begin{frame}[fragile]

\frametitle{Type Conformity: Example (1/3)}

Consider the types of some waste bins, which we can pick at:

\begin{center}
\begin{minipage}{0.4\textwidth}
\begin{lstlisting}
typeobject AnyBin
  op Pick -> [Any]
end AnyBin
\end{lstlisting}
\end{minipage}\quad%
\begin{minipage}{0.4\textwidth}
\begin{lstlisting}
typeobject PaperBin
  op Pick -> [Paper]
end PaperBin
\end{lstlisting}
\end{minipage}
\end{center}

Now, imagine being a waste picker\footnote{Waste-picking is an
admirable profession for an autonomous drone}:

\begin{itemize}

\item If you accept any trash (i.e., \texttt{AnyBin}s), then you are
also willing accept specialized trash (e.g., \texttt{PaperBin}s).

\item If you only accept specialized trash (e.g., \texttt{PaperBin}s),
then you are not willing to accept any trash (i.e., \texttt{AnyBin}s).

\end{itemize}

Hence, \texttt{PaperBin} conforms to \texttt{AnyBin}, but not
vice-versa.

\end{frame}


\begin{frame}[fragile]

\frametitle{Type Conformity: Example (2/3)}

Now, instead consider bins we can throw something into:

\begin{center}
\begin{minipage}{0.4\textwidth}
\begin{lstlisting}
typeobject AnyBin
  op Throw[Any]
end AnyBin
\end{lstlisting}
\end{minipage}\quad%
\begin{minipage}{0.4\textwidth}
\begin{lstlisting}
typeobject PaperBin
  op Throw[Paper]
end PaperBin
\end{lstlisting}
\end{minipage}
\end{center}

\begin{itemize}

\item A bin that accepts anything (i.e., \texttt{AnyBin}), can\\ also
act as a specialized bin (e.g., \texttt{PaperBin}).

\item A specialized bin however (e.g., \texttt{PaperBin}), cannot\\
act as a bin for anything (i.e., \texttt{AnyBin}).

\end{itemize}

Hence, \texttt{AnyBin} conforms to \texttt{PaperBin}, but not vice-versa.

\end{frame}

\begin{frame}[fragile]

\frametitle{Type Conformity: Example (3/3)}

Combining the two examples however, yields non-conforming bins:

\begin{center}
\begin{minipage}{0.4\textwidth}
\begin{lstlisting}
typeobject AnyBin
  op Pick -> [Any]
  op Throw[Any]
end AnyBin
\end{lstlisting}
\end{minipage}\quad%
\begin{minipage}{0.4\textwidth}
\begin{lstlisting}
typeobject PaperBin
  op Pick -> [Paper]
  op Throw[Paper]
end PaperBin
\end{lstlisting}
\end{minipage}
\end{center}

This makes sense:

\begin{itemize}

\item You cannot throw anything into a \texttt{PaperBin},\\ so it
cannot act as an \texttt{AnyBin}.

\item You can throw anything into an \texttt{AnyBin}, so it cannot act
as a \texttt{PaperBin}, from which you only ever want to pick
\texttt{Paper}.

\end{itemize}

Hence, neither \texttt{AnyBin} conforms to \texttt{PaperBin}, nor
vice-versa.

\end{frame}
