\begin{frame}

\frametitle{Principle: Everything Is an Object\footnote{Well, almost
everything}}

\begin{itemize}

\item Basic types (integers, booleans, strings, etc.) are objects

\item Classes are objects (in Emerald, mere syntactic sugar)

\item Types are objects (of a special built-in type, \texttt{Signature})

\item Language constructs however, are \textbf{\underline{not}}
objects \\ (e.g., declarations, if-statements, for-loops, programs)

\end{itemize}

\vspace{\fill}

\textbf{Alternative interpretation:}

Every valid expression evaluates to an object

\vspace{\fill}

Consequently:

\begin{itemize}

\item Type names and declarations are expressions

\item Class names and declarations are expressions

\end{itemize}

\end{frame}

% \begin{frame}
% 
% \frametitle{How Can You Test If Something Is An Object?}
% 
% \begin{itemize}
% 
% \item 
% 
% \end{itemize}
% 
% \end{frame}
% 
% 
% \begin{frame}
% 
% \frametitle{Classless Objects}
% 
% \begin{itemize}
% 
% \item In Emerald, defining an object type, and instantiting an object
% of that type are done in one go.
% 
% \end{itemize}
% 
% \end{frame}
