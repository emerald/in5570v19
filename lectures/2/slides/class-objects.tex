\begin{frame}[fragile]

\frametitle{What Is A Class (in Emerald) Anyway?}

\begin{center}

A class declares (1) an object type, and \\ (2) a means to create
instances of that type

\end{center}

\vspace{\fill}

Consequently, an Emerald class \texttt{C} is
\textbf{\underline{syntactic sugar}}\\ for an Emerald object exporting
the following methods:

\begin{lstlisting}
getSignature -> Signature
create [p1, p2, ...] -> C
\end{lstlisting}

where

\begin{itemize}

\item \lstinline{Signature} is a built-in type of all type objects

\item The value (object) returned by \texttt{create} will ``conform
to''\\ the signature returned by \texttt{getSignature}

\end{itemize}

More on type objects and conformity after an example

\end{frame}

\begin{frame}[fragile]

\frametitle{A More Elaborate (Class) Object}

The class from before, without syntactic sugar:

\begin{lstlisting}
const rand <- object RandCreator
  const RandType <- typeobject RandType
    op next -> [seed : Integer]
  end RandType
  export function getSignature -> [r : Signature]
    r <- RandType
  end getSignature
  export op create -> [r : RandType]
    r <- object Rand
      var seed : Integer <- 123456789
      const a <- 1103515245
      const c <- 12345
      const m <- 2147483648
      export operation next[] -> [r : Integer]
        seed <- (a * seed + c) # m
        r <- seed
      end next
    end Rand
  end create
end RandCreator
\end{lstlisting}

\end{frame}

