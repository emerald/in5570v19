\begin{frame}

\frametitle{Polymorphism in Emerald}

\begin{center}

In a language with \textbf{polymorphism},\\ we can write code that
works for many datatypes,\\ not just one particular type of data

\end{center}

\begin{itemize}

\item Emerald supports \textbf{inclusion polymorphism}, due to its
\emph{conformance relation}: In place of a particular type, Emerald
hapily accepts a value of a different, but conforming type

\begin{itemize}

\item You know about conformance from previous weeks

\end{itemize}

\item Emerald also supports \textbf{parametric polymorphism}, where
types depend on the actual parameters (e.g.,\\ the parameters to a
method call)

\begin{itemize}

\item See also Section 7.4 on pages 18-19 of \cite{report1991}

\end{itemize}

\end{itemize}

\end{frame}

\begin{frame}[fragile]

\frametitle{Passing the Type Parameter \underline{Implicitly}}

\inputsrc{implicit.m}

\begin{lstlisting}
$ emx implicit.x
integertype
\end{lstlisting}

\begin{itemize}

\item The \lstinline{forall} clause introduces an \emph{unconstrained
type variable}

\item We can then (somewhat backwards) use \lstinline{t} in the
signature preceeding the \lstinline{forall} clause
(\lstinline{op showType[a : t]})

\item We must use a \lstinline{forall} clause, as otherwsie
\lstinline{t} is undefined \frownie{}

\item \lstinline{t} gets the type \lstinline{ConcreteType};
\lstinline{t} can be inspected at runtime

\end{itemize}

\end{frame}

\begin{frame}[fragile]

\frametitle{There Are No Runtime Costs}

\vspace{\fill}

\begin{center}
All types are determined at compile-time!
\end{center}

\vspace{\fill}

\begin{itemize}

\item Watch out for ``type must be manifest'' errors from the Emerald
compiler; if you get these, it means that the type of some expression
cannot be determined at compile-time

\end{itemize}

\end{frame}

\begin{frame}[fragile]

\frametitle{Passing the Type Parameter \underline{Explicitly}}

\inputsrc{explicit.m}

\begin{lstlisting}
$ emx explicit.x
integertype
\end{lstlisting}

\begin{itemize}

\item Recall that, in Emerald, types are also objects

\item As another example, recall how you must explicitly pass a type
to the \lstinline{Array.of} constructor, to get an \lstinline{Array}
of that type

\item Unfortunately, we can't do much with \lstinline{t} directly (see
line 3)

\item Values of type \lstinline{Type} are assumed to only be available
at runtime

\end{itemize}

\end{frame}

\begin{frame}[fragile]

\frametitle{A Brief Interlude: \lstinline{typeof} vs. \lstinline{syntactictypeof}}

Quite unlike many popular languages, Emerald provides two ways to ask
for the type of an expression --- \lstinline{typeof} and
\lstinline{syntactictypeof}:

\begin{itemize}

\item \lstinline{typeof} gives the actual type at \emph{runtime}

\item \lstinline{syntactictypeof} gives the type determined at \emph{compile time}

\end{itemize}

\begin{center}

The Emerald system guarantees that the runtime type of an expression
will conform to its compile-time type.

\end{center}

\end{frame}

\begin{frame}[fragile]

\frametitle{\lstinline{typeof} vs. \lstinline{syntactictypeof} Illustrated}

\vspace{\fill}

What happens if we ask for \lstinline{typeof t} instead of
\lstinline{typeof a} above?

\begin{lstlisting}[language=bash]
$ diff explicit.m typeof.m
3c3
<     stdout.putstring[(typeof a)$name || "\n"]
---
>     stdout.putstring[(typeof t)$name || "\n"]
$ emx typeof.x
pat
\end{lstlisting}

\vspace{\fill}

What about \lstinline{syntactictypeof t}?

\begin{lstlisting}[language=bash]
$ diff explicit.m syntactictypeof.m
3c3
<     stdout.putstring[(typeof a)$name || "\n"]
---
>     stdout.putstring[(syntactictypeof t)$name || "\n"]
$ emx syntactictypeof.x
typetype
\end{lstlisting}

\end{frame}

\begin{frame}

\frametitle{Constraining Type Variables Such That \ldots}

\inputsrclines{replicate.m}{1}{8}

\begin{itemize}

\item Use a \lstinline{suchthat} clause

\item \lstinline{*>} means \emph{conforms to}, and the expression on
the right-hand side can be any type-valued expression

\end{itemize}

\end{frame}

\begin{frame}

\frametitle{Building Values with Types Such That \ldots}

\inputsrclines{replicate.m}{10}{21}

\begin{itemize}

\item \lstinline{t} has to be available at compile-time

\item Unfortunately, the only way to do so is with a \lstinline{forall} clause

\item Parametric polymorphism can be quite verbose in Emerald \frownie{}

\end{itemize}

\end{frame}

\begin{frame}

\frametitle{Relicating Integers and Strings}

\inputsrclines{replicate.m}{23}{40}

\end{frame}

\begin{frame}

\frametitle{Constructing Dependent Types}

\inputsrc{replicas.m}

\begin{itemize}

\item Use a \lstinline{where} clause

\item The type \lstinline{rt} \emph{depends on} the given type
\lstinline{t}

\item Constructing a value of this particular type however, is even
more tricky than for \texttt{RType}; that is, without resorting to
type assertions (\lstinline{view ... as ...})

\end{itemize}

\end{frame}
