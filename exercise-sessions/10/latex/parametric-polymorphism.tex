\begin{frame}

\frametitle{Parametric Polymorphism (in Emerald)}

\begin{itemize}

\item A language that supports \emph{polymorphism}, enables us to
write code that will work for a collection of datatypes, rather than
one particular data type.

\item Emerald supports \emph{inclusion polymorphism}, due to its
conformance relation: In place of a particular type, Emerald hapily
accepts a value of a different, but conforming type.

\item Emerald also supports \emph{parametric polymorphism}, where no
type is actually specified. Instead, a type is passed (implicitly, or
explicitly) as a parameter, and can variously constrained.

\item See also Section 7.4 on pages 18-19 of \cite{report1991}.

\end{itemize}

\end{frame}

\begin{frame}[fragile]

\frametitle{Passing the Type Parameter \underline{Explicitly}}

\inputsrc{explicit.m}

\begin{itemize}

\item Unfortunately, we cannot inspect \lstinline{t} directly; values
of type \lstinline{Type}, are rather useless at runtime --- we must
resort to asking for the type of \lstinline{a} itself

\item As another exmaple, recall how you must explicitly pass a type
to the \lstinline{Array.of} constructor, to get an \lstinline{Array}
of that type

\end{itemize}

\end{frame}

\begin{frame}[fragile]

\frametitle{Passing the Type Parameter \underline{Implicitly}}

\inputsrc{implicit.m}

\begin{itemize}

\item The \lstinline{forall} clause introduces an \emph{unconstrained
type variable}

\item We can then (somewhat backwards) use \lstinline{t} in the
signature above the \lstinline{forall} clause

\item However, \lstinline{t} now also gets the type
\lstinline{ConcreteType}, which means that the type variable is
actually useful at runtime!

\item We must use a \lstinline{forall} clause, as otherwsie
\lstinline{t} is undefined.

\end{itemize}

\end{frame}

\begin{frame}[fragile]

\frametitle{What if you don't use \lstinline{t}?}

\inputsrc{whocares.m}

\begin{itemize}

\item Who cares what type \lstinline{t} refers to!

\end{itemize}

\end{frame}

\begin{frame}

\frametitle{Constraining Type Variables Such That \ldots}

\lstinputlisting[lastline=8]{../emerald/replicate.m}

\begin{itemize}

\item Use a \lstinline{suchthat} clause

\end{itemize}

\end{frame}

\begin{frame}

\frametitle{Constructing Dependent Types}

\lstinputlisting[lastline=9]{../emerald/replicate.m}

\begin{itemize}

\item Use a \lstinline{where} clause

\end{itemize}

\end{frame}
